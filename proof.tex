\documentclass{article}
\usepackage{amsmath}
\usepackage{amsthm}
\usepackage{amssymb}
\usepackage{amsfonts}
\usepackage{physics}
\usepackage{xpatch}

\theoremstyle{plain}
\newtheorem*{theorem*}{Show}
\newtheorem{theorem}{Show}

\makeatletter
\xpatchcmd{\@thm}{\thm@headpunct{.}}{\thm@headpunct{}}{}{}
\makeatother

\title{The factorizations of square of cosecant function}
\author{WeiHua Li}
\date{\today}

\begin{document}
\maketitle

\begin{theorem*}
\begin{equation} \label{eq:0}
\sum_{k=-\infty}^{+\infty} \frac{1}{(k\pi + \frac{w}{2})^2} = \csc^2 \frac{w}{2}  \; (w \ne 2n\pi, n\in \mathbb{Z})
\end{equation}
\end{theorem*}

\begin{proof}

Gamma function is defined as
\begin{equation}
   \Gamma(z) = \int_{0}^{\infty} t^{z-1} e^{-t} \dd t,\quad \mathfrak{R}(z) > 0.
\end{equation}

Digamma function is defined as

\begin{equation}
   \psi(z) = \frac{\dd}{\dd z} \ln\Gamma(z) = \frac{\Gamma^{'}(z)}{\Gamma(z)}.
\end{equation}

We begin with Euler's reflection formula
\begin{equation}
    \Gamma(z) \Gamma(1-z) = \frac{\pi}{\sin\pi z}, \quad z \notin \mathbb{Z}
\end{equation}

Then
\begin{align*}
\ln (\Gamma(z) \Gamma(1-z)) &= \ln(\frac{\pi}{\sin \pi z}) \\
\ln\Gamma(z) + \ln\Gamma(1-z) &= \ln\pi - \ln(\sin(\pi z)) \\
\frac{\Gamma^{'}(z)}{\Gamma(z)} - \frac{\Gamma^{'}(1-z)}{\Gamma(1-z)}
&=  - \pi \frac{\cos(\pi z)}{\sin(\pi z)} = -\pi \cot\pi z \\
\psi(z) - \psi(1-z) &= -\pi \cot \pi z \\
\psi(1-z) - \psi(z) &= \pi \cot \pi z
\end{align*}

The derivative of above equation is
\begin{align*}
- \psi^{(1)}(1-z) - \psi^{(1)}(z) &= \pi \frac{\dd}{\dd z} \cot \pi z \\
\psi^{(1)}(z) + \psi^{(1)}(1-z) &= \pi^2 \csc^2 \pi z
\end{align*}


Weierstrass's definition of gamma function is
\begin{equation}
\Gamma(z) = \frac{e^{-\gamma z}}{z}  \prod_{n=1}^{\infty} \left(1 + \frac{z}{n} \right)^{-1} e^{z/n} .
\end{equation}

Thus,
\begin{align*}
\psi(z) &= \frac{\dd}{\dd z} \ln\Gamma(z) \\
&= \frac{\dd}{\dd z} \left(  -\gamma z - \ln(z) + 
\sum_{n=1}^{\infty} \left(-\ln(1+\frac{z}{n}) + \frac{z}{n}\right)  \right) \\
&= -\gamma - \frac{1}{z} + \sum_{n=1}^{\infty} \left( \frac{1}{n} - \frac{1}{n+z} \right) .
\end{align*}

The derivative of digamma function is
\begin{equation}
\psi^{(1)}(z) = \frac{1}{z^2} + \sum_{n=1}^{\infty} \frac{1}{(n+z)^2} = \sum_{n=0}^{\infty} \frac{1}{(n+z)^2}.
\end{equation}

Thus,
\begin{align*}
\sum_{k=0}^{\infty} \frac{1}{(z+k)^2} + \sum_{k=0}^{\infty} \frac{1}{(1-z+k)^2} &= \pi^2 \csc^2 \pi z \\
\sum_{k=0}^{\infty} \frac{1}{(z+k)^2} + \sum_{n=-\infty}^{-1} \frac{1}{(n+z)^2} &= \pi^2 \csc^2 \pi z \\
\sum_{k=-\infty}^{\infty} \frac{1}{(z+k)^2} &= \pi^2 \csc^2 \pi z \\
\end{align*}

Then set $z = w / (2\pi)$, equation (\ref{eq:0}) is proofed.
\end{proof}

\end{document}
